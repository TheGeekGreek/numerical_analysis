\section{Extrapolation and Step Size Control}

\subsection{The Implicit Trapezoidal Rule}
The \index{implicit trapezoidal rule}\emph{implicit trapezoidal rule} is given by

\begin{equation}
	y_1 = y_0 + \frac{h}{2}\left( f(x_0,y_0) + f(x_1,y_1) \right)
\end{equation}

As one observes, the discretization $y_1$ of $y(x_1)$ is calculated implicitely. There are two simple approaches to handle this impliciteness. First of all, $y_1$ is a fixed point of the function

\begin{equation}
	\Phi(y;x_0,y_0,h) := y_0 + \frac{h}{2}\left( f(x_0,y_0) + f(x_0 + h,y) \right)
\end{equation}

Hence we may apply a simple \index{fixed point iteration}\emph{fixed point iteration} of the form

\begin{equation}
	y_1^{(k + 1)} = \Phi\left(y_1^{(k)};x_0,y_0,h\right) \qquad k = 0,1,2,\hdots
\end{equation}

until a certain tolerance or a user-specified maximal number of iterations is reached. An implementation of the fixed point iteration and the implicit trapezoidal rule can be found in listing \ref{lst:fixediter} and \ref{lst:ITR} respectively.

\begin{listing}
	\m{src/fixediter.m}
	\caption{Fixed point iteration.}
	\label{lst:fixediter}
\end{listing}

\begin{listing}
	\m{src/ITR.m}
	\caption{Implicit trapezoidal rule.}
	\label{lst:ITR}
\end{listing}

The second approach would be applying the \index{Newton iteration}\emph{Newton iteration}

\begin{equation}
	y_1^{(k + 1)} = y_0^{(k)} - \left( D\Phi\left(y_1^{(k)};x_0,y_0,h\right)\right)^{-1}\Phi\left(y_1^{(k)};x_0,y_0,h\right) \qquad k = 0,1,2,\hdots
\end{equation}
